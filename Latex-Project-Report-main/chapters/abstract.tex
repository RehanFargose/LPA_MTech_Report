\chapter*{Abstract}
\addcontentsline{toc}{chapter}{Abstract}

In today’s world, due to the ever-rising population and inability of the economy to cope up with it has led to a rise in crimes, law violations, struggle for resources, etc. One of the main concerns is the massive backlog in the Indian courts, pertaining to both the population and the inefficiency of the Indian Judiciary. Legal Professionals such as Judges, Lawyers, Consultants often must go through past court documents in order to come across “Precedents” (Past Rulings) which can help in passing informed judgements in Ongoing trials. This is a tedious process and can take months if not years, leading to delay in Judgements and as the proverb goes “Justice delayed is Justice denied.” 
NLP (Natural Language Processing) is a branch of ML (Machine Learning) that deals with making machines comprehend Human language and its context. By feeding aforementioned models with Court Datasets, we can train them to not only find Legal Precedents but can also be used to Summarize and Predict the actual verdict. 
In this project, we aim to create a Legal Precedent Assistant using NLP that will aid in the process of finding Precedents and predicting verdicts based on presented evidence to reduce the burden of the Indian Judiciary and improve its efficiency. This project has 3 main goals: Mapping the IPC codes found in a Court Document, Using the Mapped IPC codes alongside the context in the document to predict an outcome(Appellant/Defendant Wins) coupled with the Judgment And finally Summarizing the entire document and Judgement passed in Simplified terms by removing Legal Jargon, thus allowing General Public to understand the Court’s Proceedings. The performance of the model was found to be accurate 85\% of the time but can be improved further with larger models and more refined Datasets.


\textbf{Keywords:} Transformers, Encoders, BERT, LegalBERT, Llama, DAPT, Verdict Prediction, Summarization, Case Law, Legal Precedents, IPC Mapping
